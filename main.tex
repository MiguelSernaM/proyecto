\documentclass{article}
\usepackage[utf8]{inputenc}
\usepackage[spanish]{babel}
\usepackage{listings}
\usepackage{graphicx}
\graphicspath{ {images/} }
\usepackage{cite}

\begin{document}

\begin{titlepage}
    \begin{center}
        \vspace*{1cm}
            
        \Huge
        \textbf{Memoria del computador}
            
        \vspace{0.5cm}
        \LARGE
        Taller
            
        \vspace{1.5cm}
            
        \textbf{Miguel Angel Serna Montoya}
            
        \vfill
            
        \vspace{0.8cm}
            
        \Large
        Despartamento de Ingeniería Electrónica y Telecomunicaciones\\
        Universidad de Antioquia\\
        Medellín\\
        Septiembre de 2020
            
    \end{center}
\end{titlepage}

\tableofcontents

\section{Introducción} \label{Introducción}
Esta es la primera sección, podemos agregar algunos elementos adicionales y todo será escrito correctamente. Más aún, si una palabra es demasiado larga y tiene que ser truncada, babel tratará de truncarla correctamente dependiendo del idioma.

\section{Contenido} \label{Contenido}

Esta sección es para ver qué pasa con los comandos 
que definen texto

El paquete también agrega un comportamiento especial 
a <<estas marcas para hacer citas textuales>> tal como 
lo indican las reglas de la RAE. \cite{dirac}



A continuación se presenta el logo de C++ Figura (\ref{fig:cpplogo})

\begin{figure}[h]
\includegraphics[width=4cm]{cpplogo.png}
\centering
\caption{Logo de C++}
\label{fig:cpplogo}
\end{figure}
\section{La Memoria del computador}
\subsection{¿Qué es la memoria del computador?}
La memoria es un componente esencial en los computadores, es la herramienta que me permite sacar información del disco duro, almacenarla temporalmente y asi facilitar el procesamiento de datos e instrucciones que ejecuta la cpu y una vez que estas fueron concretadas regresa la información obtenida a el disco duro reemplazando los archivos que fueron extraidos
\subsection{Tipos de memoria}

\subsection{¿Como se gestiona la memoria en un computador?}
\subsection{¿Qué hace que una memoria sea más rapida que otra? ¿Por qué es importante?}

En la sección de teoremas (\ref{Contenido})

\section{Conclusión} \label{Conclusión}

\bibliographystyle{IEEEtran}
\bibliography{references}

\end{document}
